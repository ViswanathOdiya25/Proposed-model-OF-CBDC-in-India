%%%%%%%%%%%%%%%%%%%%%%%%%%%%%%%%%%%%%%%%%%%%%%%%%%%%%%%%%%%%%%%%%%%%%%%%%%%%
%% Chapter 3
%%Indian Institute of Information Technology Kalyani
%% All rights are reserved.
%%%%%%%%%%%%%%%%%%%%%%%%%%%%%%%%%%%%%%%%%%%%%%%%%%%%%%%%%%%%%%%%%%%%%%%%%%%%
%
\chapter{Problem Statement and Scope}
\label{chp3}

\section{Introduction}
\label{chp3.introduction}

As India rapidly moves toward a digitally empowered economy, its financial infrastructure is evolving at an unprecedented pace. Traditional payment systems—such as NEFT, RTGS, and even UPI—have played a pivotal role in enabling seamless digital transactions. However, these systems are now encountering significant challenges related to \textbf{scalability, user privacy, cybersecurity, and equitable financial inclusion}. With the growing volume and complexity of digital transactions, concerns around system overload, data protection, and access gaps for underserved populations have become increasingly prominent.

At the same time, the \textbf{global rise of cryptocurrencies and decentralized finance (DeFi)} has drawn attention to the limitations of existing monetary systems. These developments have sparked important debates around \textbf{monetary sovereignty, regulatory oversight, and the role of central banks in the digital era}. The decentralized nature of cryptocurrencies presents both an opportunity and a risk—highlighting the need for a secure, state-backed digital alternative that can offer the innovation of crypto with the stability of fiat money.

In this context, \textbf{Central Bank Digital Currency (CBDC)} emerges as a potential solution that bridges the gap between traditional fiat currencies and emerging digital payment paradigms. A well-designed CBDC has the potential to enhance financial inclusion, improve transaction efficiency, and strengthen the resilience and sovereignty of a nation’s monetary system.

This chapter explores the \textbf{problem landscape} in detail and establishes the \textbf{scope and direction} of this study. It seeks to understand why there is an urgent need for a digital public currency in India, what gaps currently exist in the financial system, and how a CBDC—particularly one that can integrate or interoperate with existing platforms like UPI—might address these challenges in a practical and secure manner.

 
\section{Limitations of Current Payment Systems}
\label{chp3.limitations}

While systems like UPI, RTGS, NEFT, and IMPS have significantly improved transaction speed and accessibility, they remain reliant on continuous internet connectivity, centralized clearing infrastructure, and bank-centric access models. The major limitations include:

\begin{itemize}
    \item \textbf{Dependency on banking hours and infrastructure} in some cases (e.g., NEFT).
    \item \textbf{Limited offline functionality}—no payment system is usable without a network connection.
    \item \textbf{Intermediary costs} and settlement delays for certain transaction types.
    \item \textbf{Limited financial reach} to the unbanked and underbanked population.
\end{itemize}

\section{Challenges with Cash and Private Cryptocurrencies}
\label{chp3.challenges}

\subsection*{Cash}
Though cash is being used tradionally for a really long period of time, and is  widely accepted and anonymous, it poses certain risks such as:

\begin{itemize}
    \item High costs of printing, transportation, and secure storage.
    \item Difficulties in traceability, enabling black money and tax evasion.
    \item Counterfeit currency circulation that undermines trust and economic stability.
\end{itemize}

\subsection*{Private Cryptocurrencies}
Private digital currencies such as Bitcoin and Ethereum, though they are innovative but they still come with multiple concerns and challenges such as:

\begin{itemize}
    \item \textbf{Price volatility} and speculative use rather than as a stable medium of exchange.
    \item \textbf{Lack of regulation} and control, making them risky for national economies.
    \item \textbf{Anonymity} makes them susceptible to misuse in illegal activities.
\end{itemize}

\section{Problem Statement}
\label{chp3.statement}

The current Indian monetary and payment infrastructure lacks a fully sovereign, secure, and digitally native currency that is:

\begin{itemize}
    \item Usable both online and offline;
    \item Accessible to both banked and unbanked populations;
    \item Resistant to counterfeiting;
    \item Capable of supporting programmable features for advanced use cases;
    \item Interoperable with existing digital payment rails such as UPI;
    \item Privacy-respecting yet compliant with regulatory frameworks.
\end{itemize}

\textbf{Hence, there is a need to design a robust CBDC model that can addresse these limitations while also aligning with present India’s unique economic, infrastructural, and social landscape.}

\section{Scope of the Study}
\label{chp3.scope}

This thesis focuses on proposing a model for a Central Bank Digital Currency (CBDC) tailored for the Indian economy, with specific attention to:

\begin{itemize}
    \item Evaluating global CBDC architectures and adapting relevant features for India.
    \item Designing a hybrid ledger model with both account-based and token-based functionalities.
    \item Ensuring offline transaction capability using technologies like NFC and QR codes.
    \item Incorporating a tiered-KYC approach to balance privacy and compliance.
    \item Proposing interoperability with existing digital systems like UPI and e-wallets.
    \item Analyzing regulatory frameworks that would impact the design and deployment of a CBDC in India.
\end{itemize}

The study does not attempt to define monetary policy impacts in detail but will highlight potential implications and policy considerations.

\section{Summary}
\label{chp3.summary}

In this chapter, we looked at the main challenges facing India’s current payment and monetary systems—from issues like scalability and privacy to the heavy reliance on cash and the rise of private cryptocurrencies. These challenges make it clear that India needs a digital currency that’s not just secure and sovereign, but also easy to use across the country’s diverse financial landscape. It should work smoothly even offline, protect user privacy, and fit well with popular systems like UPI.

With these points in mind, the next chapter will introduce a detailed technical and functional design for a CBDC made specifically for India. This design aims to blend the strengths of modern technology with the country’s unique financial needs, ensuring that the digital rupee can serve everyone efficiently and safely in the years to come.

 