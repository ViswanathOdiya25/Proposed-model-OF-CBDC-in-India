 \cleardoublepage
 \pagenumbering{roman}
 \setcounter{page}{5}
 \begin{center}
 {\LARGE \textsc{Abstract}}
 \end{center}
 \vspace{1cm}
 
As digital transactions become more common and the use of cash continues to decline, the Reserve Bank of India (RBI) has taken steps to explore a new form of currency - the Central Bank Digital Currency (CBDC), also known as the Digital Rupee. This thesis looks into the reasons behind India's move toward a digital currency and examines how it could improve payment systems, lower currency management costs, and support financial inclusion, especially in areas where access to traditional banking is limited.

Drawing from global case studies, the research identifies challenges and opportunities unique to the Indian context. It then proposes a model for implementing CBDC in India, built around a two-tier system where the RBI works with licensed financial institutions to distribute the digital currency. The model focuses on making the system accessible, secure, and compatible with platforms such as UPI that are already widely used in India.

The design also includes features such as off-line payments and a layered KYC (Know Your Customer) approach, which aims to protect user privacy while meeting regulatory requirements. In general, the proposed model is evaluated for its scalability, security, and potential to strengthen India’s push toward a digital economy.\\
 \vspace{0.5cm} 
 
 \noindent \textbf{Keywords:}~Central Bank Digital Currency (CBDC), Digital Rupee, Reserve Bank of India (RBI), Financial Inclusion, Digital Payments, Two-Tier Architecture, KYC, UPI, India, Offline Transactions
5
