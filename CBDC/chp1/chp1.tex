%%%%%%%%%%%%%%%%%%%%%%%%%%%%%%%%%%%%%%%%%%%%%%%%%%%%%%%%%%%%%%%%%%%%%%%%%%%%
%% Chapter 1
%%Indian Institute of Information Technology Kalyani
%% All rights are reserved.
%%%%%%%%%%%%%%%%%%%%%%%%%%%%%%%%%%%%%%%%%%%%%%%%%%%%%%%%%%%%%%%%%%%%%%%%%%%%
%


\chapter[Introduction]{Introduction}
\label{chp1}
%\setcounter{minitocdepth}{1}
%\minitoc
\section{Introduction}
\label{chp1.1}
In this era of the Internet, we are surrounded by smart devices connected through extensive digital networks that bring most services a click away. With this new surge in digital payments, the traditional use of cash has been declining rapidly \cite{bohme2015bitcoin}. Suddenly, Cryptocurrencies such as Bitcoin have their popularity increase through the roof \cite{narayana2016bitcoin}. This has further prompted the central banks of different nations to explore the possibility of having their own digital currencies \cite{barontini2020}. This new form of money that can be used by the public for transactions are referred as \textbf{Central Bank Digital Currencies (CBDCs)}.

Where cryptocurrencies are completely decentralized, CBDC's are controlled by the Central Bank which implement CBDC's using centralized ledgers \cite{garratt2020} with people having their own deposit accounts, quite different from the modern Bitcoin which rely on decentralized technology such as blockchain \cite{dyson2016}.

Traditionally, central banks issue two forms of money: \textit{physical cash} \cite{kumhof2015},  which is accessible directly by the public, and \textit{reserve deposits}, which are held by commercial banks at the central bank. Although individuals can withdraw cash from their commercial bank accounts, they cannot directly access central bank reserves. These reserves are used by commercial banks to issue loans which indirectly provide central bank money to the public. A CBDC would offer a third form of liability \cite{bains2022}. 

The introduction of a CBDC would provide the public with a third form of direct access to central bank-issued money \cite{mancini2018}. Consequently, CBDCs would coexist with cash and retail bank deposits, potentially reshaping the structure of the monetary system. As individuals would have the freedom to choose between cash, CBDC, and traditional deposits, the issuance of a CBDC introduces a competitive element in the forms of money held by the public 
 \cite{grym2017}.
%%%%%%%%%%%%%%%%%%%%%%%%%%%%%%%%%%%%%%%%%%%%%%%%%%%%%%%%%%%%%%%%%%%%%%%%%%%%%%%%%%%%%%%
%%%%%%%%%%%%%%%%%%%%%%%%%%%%%%%%%%%%%%%%%%%%%%%%%%%%%%%%%%%%%%%%%%%%%%%%%%%%%%%%%%%%%%%

%%%%%%%%%%%%%%%%%%%%%%%%%%%%%%%%%%%%%%%%%%%%%%%%%%%%%%%%%%%%%%%%%%%%%%%%%%%%%%%%%%%%%%%
%%%%%%%%%%%%%%%%%%%%%%%%%%%%%%%%%%%%%%%%%%%%%%%%%%%%%%%%%%%%%%%%%%%%%%%%%%%%%%%%%%%%%%%


\section{What is CBDC?}
\label{chp1.2}

A Central Bank Digital Currency (CBDC) is a currency in digital form which is issued by the central bank \cite{boar2021}. This currency can be used for money transactions just like any other mode of currency such as hard money or online transactions.
For normal people CBDC can be said as the digital equivalent of the hard money that has been used traditionally(eg. paper notes, coins).

These definitions show that a CBDC is a liability of the issuing central bank and is different from cash in its physical attributes, even though a CBDC serves the same function as cash \cite{mengle2018}.


\section{Importance of CBDC in India}
\label{chp1.3}

India is a vast country with a large population. The number goes even higher when we take into account the number of transactions that happens in a day in India. With increase in digital transactions in India, especially through platforms like UPI.

This rapid digitization witnessed in India through various Digital India initiatives, CBDC would be the next natural step for Digital India \cite{rbi2023}. This growth highlights the country's readiness for CBDC integration \cite{bains2022, agarwal2021}.  While creating an ecosystem for true digital monetary ecosystem and also reducing the dependency on physical cash.

Introducing CBDC can solve a lot of problems which are often related with traditional currency such as:
\begin{itemize}

    \item \textbf{Reduction of Handling costs of Cash:} Hard money often comes with high costs due to printing, transportation, storage and security. Having a digital alternative can significantly reduce these expenses \cite{klein2020}.

    \item \textbf{Financial Inclusion:} A CBDC can extend financial services to the unbanked and underbanked population through mobile wallets and offline functionalities, especially in rural and semi-urban regions \cite{catalini2021,agarwal2021}.

    \item \textbf{Counterfieting of physical money:} Hard or Physical money always comes with a chance of being counterfiet, while CBDC is cryptographically secured and can be identified uniquely. Thus, eleminating the risk of counterfiet money \cite{rbi2023}.

    \item \textbf{Regulated by Government:} Unlike widely-known cryptocurrencies which are volatile, CBDC are government-issued digital currency making it secure and regulated by the government, preserving monetary sovereignty.

    \item \textbf{Cross-Border Payments:} CBDCs have the potential to streamline international remittances, making them faster, more transparent, and cost-effective \cite{bis2021}.

    \item \textbf{Targeted Subsidy Delivery:} Programmable CBDCs can enable efficient, conditional delivery of subsidies, minimizing leakages and ensuring proper utilization.

    \item \textbf{Resilience and Accessibility:} Offline CBDC features can ensure payment accessibility during internet outages, enhancing the robustness of the payment ecosystem \cite{auer2020}.
\end{itemize}



\section{Motivations}
India is slowly moving towards a nation with digitally driven economy. This increase in digital transactions via platforms like UPI, as the usage of smaartphones and the internet increases, and with increase in population of cryptocurrencies have raises the question about sovereign money \cite{barontini2020}.

Even though private cryptocurrencies has their own secure network and technological innovation. They still pose the challenges of being volatile, limmited regulation and has potential threat to monetary policies.
While a CBDC becomes a great alternative that can monetize the indian monetary system.

This study is motivated by the following key factors:

\begin{itemize}
    \item \textbf{To design a CBDC model tailored to India's needs,} considering its large population, varying levels of financial literacy, and infrastructural diversity.

    \item \textbf{To overcome limitations in the existing payment systems,} such as inefficiencies in cross-border transactions, cash dependence, and digital divide.

    \item \textbf{To contribute to policy and academic discourse} by proposing a feasible and technically sound framework for CBDC implementation in India.

    \item \textbf{To align with RBI’s ongoing pilot programs} and build on the early-stage exploration and experimentation being undertaken at the institutional level \cite{menon2022}.

    \item \textbf{To ensure national monetary sovereignty} and reduce reliance on volatile private digital assets through a centralized, programmable, and inclusive digital currency system.
\end{itemize}

To have a system that reaches the goal of creating such a model that comprise and completes the digital ecosystem for transactions is the main motive for this study of CBDC in India.


\section{Objectives}

The proposed Central Bank Digital Currency (CBDC) model aims to contribute toward the modernization of India’s financial system while ensuring inclusivity, resilience, and regulatory compliance. The specific objectives of this thesis are as follows:

\begin{itemize}
    \item \textbf{To design a hybrid CBDC architecture} that balances central control by the RBI with decentralized features to ensure operational resilience and efficiency \cite{boar2021}.

    \item \textbf{To ensure interoperability} with existing digital payment infrastructure such as UPI and  mobile wallets.

    \item \textbf{To incorporate offline transaction capability} using technologies like NFC or QR codes to support users in areas with limited internet connectivity.

    \item \textbf{To propose a privacy-preserving framework} using tiered KYC, allowing small-value anonymous transactions while maintaining compliance with financial regulations \cite{auer2020, bains2022} .

    \item \textbf{To explore a token-based mechanism} for peer-to-peer and merchant payments that mimics the anonymity and ease of cash.

    \item \textbf{To support programmability features} for smart contracts in government-to-person (G2P) payments, targeted subsidies, or conditional transfers \cite{mengle2018, catalini2021}. 

    \item \textbf{To analyze the security mechanisms} that mitigate risks such as double-spending, counterfeiting, and fraud.

    \item \textbf{To evaluate scalability and performance} through simulation or prototyping in real-world use cases (e.g., retail, P2P, cross-border).

    \item \textbf{To align the model with Indian regulatory frameworks} such as the IT Act, PSS Act, and data localization norms.
\end{itemize}

These objectives are geared toward developing a robust, future-ready digital currency model tailored for India. which can solve the major issues of physical currencies while adding its own benefits.


\section{Organization of the Thesis}
\label{chp1.organization}

This thesis is organized into 8 chapters, each chapter builds a comprehensive understanding of Central Bank Digital Currency (CBDC) in the Indian context and to present a robust, proposed model: 

\begin{itemize}
    \item \textbf{Introduction:} Provides a background on money’s evolution, the emergence of CBDCs, their importance for India, and the objectives and motivation behind this study.

    \item \textbf{Literature Review:} Discusses various forms of digital currencies including cryptocurrencies, stablecoins, and CBDCs. It also surveys existing CBDC implementations globally and India’s current digital payment landscape.

    \item \textbf{Problem Statement and Scope:} Highlights the limitations of current financial systems and payment mechanisms. It defines the specific problems that the proposed CBDC model aims to address.

    \item \textbf{Proposed Model for CBDC in India:} Describes the architecture of the proposed CBDC system, including technical design, privacy considerations, interoperability features, and key stakeholders.

    \item \textbf{Technical Implementation:} Presents the implementation plan, tools and frameworks used, system flow diagrams, and use cases for the proposed solution.

    \item \textbf{Benefits and Challenges:} Discusses the expected benefits of the proposed CBDC model, including increased financial inclusion and monetary control, and analyzes challenges such as privacy, cybersecurity, and infrastructure limitations.

    \item \textbf{Future Scope:} Explores potential enhancements, including integration with other national CBDCs, artificial intelligence applications, offline transaction improvements, and DeFi interoperability.

    \item \textbf{ Conclusion:} Summarizes the findings, evaluates the feasibility of the proposed model, and provides concluding remarks on the study’s relevance and potential impact.
\end{itemize}


\section{Summary}
\label{chp1.summary}

In this chapter, we described the emerging concept of Central Bank Digital Currency (CBDC) along with its increasing significance in global and Indian economics. The rise of cryptocurrencies, alongside the decline of cash used, has created the available need for a safe, monitored, and accessible digital currency.

We discussed the positive impacts of CBDC on India, particularly in relation to financial inclusion, reduction of counterfeiting, and preservation of monetary autonomy. This study was motivated and driven by the unique challenge of constructing a CBDC framework tailored to India’s digitally equipped infrastructure and socio-economic reality.

This chapter also outlined the principal goals of our model which include a guarantee of offline functionality, provision of privacy through a tiered-KYC, and seamless integration with other digital payment systems like UPI.

In conclusion, this section presented the organization of the thesis to provide a roadmap for the forthcoming chapters, which will analyze the technological, legislative, and pragmatic aspects of an Indian-centric CBDC.



